\documentclass[12pt]{article}
\usepackage[paper=letterpaper,margin=2cm]{geometry}
\usepackage{amsmath,amssymb,amsfonts}
\usepackage{newtxtext, newtxmath}
\usepackage{enumitem}
\usepackage{titling}
\usepackage[colorlinks=true]{hyperref}

\usepackage{listings}

\setlength{\droptitle}{-6em}

\begin{document}

\center
Aprendizagem 2024\\
Homework I -- Group 016\\
(ist1106022, ist1106720)\vskip 1cm

\large{\textbf{Part I}: Pen and paper}\normalsize

\begin{enumerate}[leftmargin=\labelsep]
    \item Question summary can go here.
          \begin{enumerate}
              \item Place your solution. Math can be entered using the equation
                    environment like this
                    \begin{equation}
                        \vec{\mathbf{r}} = \vec{\mathbf{r}}_{0} + \vec{\mathbf{v}}_{0}t + \frac{1}{2}\vec{\mathbf{a}}t^{2}
                    \end{equation}
                    If you then where working in say the $x$-direction and had some numbers % A percent sign allows you to comment.
                    %The dollar signs around something in a line of text is for "in-line math"
                    \begin{equation}
                        \begin{array}{r@{~=~}l}
                            x & x_{0} + v_{x0}t + \frac{1}{2}a_{x}t^{2}                                                            \\ [2ex]
                              & 1.2~\text{m} + (4.0~\text{m/s})(3.0~\text{s}) + \frac{1}{2}(-1.0~\text{m/s}^{2})(3.0~\text{s})^{2} \\ [2ex]
                              & \boxed{8.7~\text{m}}
                        \end{array}
                    \end{equation}

              \item When you get to the next part, you can add a \verb"\item" to get the appropriate label. Also,
                    if you don't like all the equation numbers, you can use the following to have the equation with
                    no number
                    \begin{equation*}
                        \sum\vec{\mathbf{F}} = m\vec{\mathbf{a}}
                    \end{equation*}

              \item For more details on putting math into {\LaTeX} documents you can see
                    \href{https://www.overleaf.com/learn/latex/Mathematical_expressions}{this page on Overleaf}.
          \end{enumerate}

    \item We you get to the next problem, you can end the enumerate for the parts of the previous problem and then add another item.
          \begin{enumerate}
              \item Use a nested enumerate environment to label the parts of the next problem.
              \item For a quick and broad overview of how to create documents in {\LaTeX} see
                    \href{https://www.overleaf.com/learn/latex/Learn_LaTeX_in_30_minutes}{this quick tutorial from Overleaf}.
          \end{enumerate}
\end{enumerate}

\large{\textbf{Part II}: Programming}\normalsize

\begin{enumerate}[leftmargin=\labelsep]
    \item Applying \texttt{f\_classif} from the \texttt{sklearn} library upon our dataset (after splitting into feature data matrix and target vector) allowed us to understand the discrimantive power of each feature. \\
    \begin{lstlisting}
        
    \end{lstlisting}

    \item
\end{enumerate}

\vskip 1cm
\textbf{End note}: do not forget to also submit your Jupyter notebook

\end{document}
